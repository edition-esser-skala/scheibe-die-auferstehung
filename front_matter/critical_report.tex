\documentclass[tocstyle=ref-genre]{ees}

\shorttitle{Die Auferstehung}

\begin{document}

\eesTitlePage

\eesCriticalReport{
  – & –  & A       & S 2 may be replaced by A throughout the piece (as indicated by \textit{Canto 2 overo Alto} in all respective movements). \\
  1  & 73  & ob 1, 2 & grace note missing in \A1 \\
  8  & 122 & vl 1    & 2nd \quarterNote\ in \A1: \sharp f′8–d′8 \\
  9  & 17  & vla     & bar in \A1: \flat b′1 \\
  10 & 62  & B       & grace note missing in \A1 \\
     & 80  & B       & grace note missing in \A1 \\
     & 164 & vla     & bar in \A1: \sharp c′4–\crotchetRest \\
  16 & 78  & vl 2    & 1st \quarterNote\ in \A1: b′4 \\
     & 82  & vla     & 2nd \quarterNote\ in \A1: a′4 \\
  19 & 131 & vl 1    & 3rd \quarterNote\ in \A1: g′4 \\
     & 156 & cor 1   & last \eighthNote\ in : g″8 \\
}

\eesToc{
\begin{movement}{gottduwirst}
  \item[Coro]
  Gott! du wirſt ſeine Seele nicht in der Hölle laſſen!\\
  Und nicht zugeben, daß dein Heiliger die Verweſung ſehe!
\end{movement}

\begin{movement}{judaeazittert}
  \item[Tenore I]
  Judäa zittert! ſeine Berge beben!\\
  der Jordan flieht den Strand!\\
  Was zitterſt du, Judäens Land?\\
  Ihr Berge, warum bebt ihr ſo?\\
  Was war dir, Jordan, daß dein Strom zurücke floß?\\
  Der Herr der Erde steigt empor aus ihrem Schooß,\\
  tritt auf den Fels, und zeigt der ſtaunenden Natur ſein Leben.\\
  Des Himmels Myriaden liegen auf der Luft rings um ihn her;\\
  und Cherub Michael fährt nieder,\\
  und rollt des vorgeworfnen Steines Laſt\\
  hinweg von ſeines Königs Gruft.\\
  Sein Antlitz flammt, ſein Auge glühet.\\
  Die Schaar der Römer ſtürzt erblaßt auf ihre Schilde:\\
  Flieht, ihr Brüder! der Götter Rache trifft uns! fliehet!
\end{movement}

\begin{movement}{meingeist}
  \item[Tenore I]
  Mein Geiſt voll Furcht und Freude bebet:\\
  der Fels zerſpringt! die Nacht wird lichte!\\
  Seht, wie er auf den Lüften ſchwebet!\\
  Seht, wie von ſeinem Angeſichte\\
  die Glorie der Gottheit ſtrahlt.\\
  Rang Jeſus nicht mit tauſend Schmerzen?\\
  Empfing ſein Gott nicht ſeine Seele?\\
  Floß nicht ſein Blut aus ſeinem Herzen?\\
  Hat nicht der Held in dieſer Hölle\\
  der Erde ſeine Schuld bezahlt?
\end{movement}

\begin{movement}{triumph}
  \item[Coro]\enlargethispage\baselineskip
  Triumph! Triumph! des Herrn Geſalbter ſieget!\\
  Er ſteigt aus ſeiner Felſengruft.\\
  Triumph! Triumph! ein Chor von Engeln flieget\\
  mit lautem Jubel durch die Luft.
\end{movement}

\begin{movement}{diefrommen}
  \item[Tenore I]
  Die frommen Töchter Zions gehn verwundernd\\
  durch des offnen Grabes Thür;\\
  und ſchaudernd fahren ſie zurück.\\
  Sie ſehn, in Glanz gehüllt, den Boten des Ewigen,\\
  der freundlich ſpricht:

  \item[Canto]
  Entſetzt euch nicht!\\
  Ich weiß, ihr suchet euren Todten,\\
  den Nazaräer Jeſus hier,\\
  daß ihr ihn ſalbt, daß ihr ihn klagt.\\
  Hier iſt er nicht vorhanden.\\
  Er hat es euch zuvor geſagt:\\
  Er lebt! Er lebt! Er iſt erſtanden!
\end{movement}

\begin{movement}{seygegruesset}
  \item[Alto]
  Sey gegrüßet, Fürſt des Lebens!\\
  Jauchzet, die ſein Tod betrübte!\\
  Er, den dieſer Hügel deckte,\\
  er, Jeſus, lebt!\\
  Ihr klagt vergebens!\\
  Sehet da, ſein leeres Grab!\\
  Der die Todten auferweckte,\\
  ſollte der im Grabe bleiben?\\
  Himmel! Soll der Gottgeliebte,\\
  ſoll der Gottheit Sohn zerſtäuben?\\
  Todesengel laſſet ab!
\end{movement}

\begin{movement}{weristdie}
  \item[Tenore I]
  Wer iſt die Sionitin, die vom Grabe ſo ſchüchtern\\
  in den Garten flieht, und weinet?\\
  Nicht lange, Jeſus ſelbſt erſcheinet,\\
  doch unerkannt, und ſpricht ihr zu:

  \item[Basso]
  O Tochter, warum weineſt du?

  \item[Alto]
  Herr, ſage, nahmſt du meinen Herrn aus dieſem Grabe?\\
  Wo liegt er? Ach vergönne, daß ich ihn hole,\\
  daß ich ihn mit Thränen netze;\\
  daß ich ihn mit dieſen Salben noch im Tode ſalben könne,\\
  wie ich im Leben ihn geſalbt.

  \item[Basso]
  Maria!

  \item[Tenore I]
  So ruft mit holder Stimm ihr Freund,\\
  in ſeiner eigenen Geſtalt:

  \item[Basso]
  Maria!

  \item[Alto]
  Mein Meiſter! ach!

  \item[Tenore I]
  Sie fällt zu ſeinen Füßen nieder,\\
  umarmt ſie, küßt ſie, weint.

  \item[Basso]
  Du ſollſt mich wieder ſehen!\\
  noch werd ich nicht zu meinem Vater gehen.\\
  Steh auf, und ſuche meine Brüder, und meinen Simon!\\
  Sag: Ich leb und will ihn ſehen.
\end{movement}

\begin{movement}{vaterdeiner}
  \item[Alto, Tenore I]
  Vater deiner ſchwachen Kinder,\\
  der Gefallne, der Betrübte\\
  hört von dir den erſten Troſt.\\
  Meiſter der gerührten Sünder,\\
  die dich ſuchte, die dich liebte,\\
  fand bey dir den erſten Troſt.\\
  Tröſter! Vater! Menſchenfreund!\\
  O wie wird durch jede Zähre\\
  dein erbarmend Herz erweicht!\\
  Sagt, wer unſrem Gotte gleicht,\\
  der die Miſſethat vergiebet?\\
  Sagt, wer unſrem Gotte gleicht,\\
  der die Miſſethäter liebet?\\
  Liebe, die du ſelbſt geweint.\\
  O wie wird durch jede Zähre\\
  dein allgütig Herz erweicht!
\end{movement}

\begin{movement}{freundinnen}
  \item[Tenore I]
  Freundinnen Jeſu!\\
  Sagt, woher ſo oft in dieſen Garten?\\
  Habt ihr nicht gehört, er lebe?\\
  Ihr zärtlichen Betrübten hofft\\
  den Göttlichen zu ſehn, den Magdalena ſah?\\
  Ihr ſeyd erhört. Urplötzlich iſt er da,\\
  und Aloen und Myrrhen duftet ſein Gewand:

  \item[Basso]
  Ich bin es! ſeyd gegrüßt!

  \item[Tenore I]
  Sie fallen zitternd nieder.\\
  Sein Arm erhebt ſie wieder:

  \item[Basso]
  Geht hin in unſer Vaterland,\\
  und ſagt den Jüngern an:\\
  Ich lebe, und fahre bald hinauf\\
  in meines Vaters Reich;\\
  doch will ich alle ſehn,\\
  bevor ich mich für euch zu meinem Gott\\
  und eurem Gott gen Himmel hebe.
\end{movement}

\begin{movement}{ichfolge}
  \item[Basso]
  Ich folge dir, erklärter Held!\\
  dir, Erſtling der entſchlafnen Frommen!\\
  Triumph! Triumph! der Tod iſt weggenommen,\\
  der Tod, der auf der Welt der Geiſter lag.\\
  Das Fleiſch, das in den Staub zerfällt,\\
  wächſt frölich aus dem Staube wieder.\\
  O ruft in Hoffnung, meine Glieder,\\
  bis an den großen Erndtetag!
\end{movement}

\begin{movement}{todwo}
  \item[Coro]
  Tod! wo iſt dein Stachel?\\
  dein Sieg, o Hölle! wo iſt er?\\
  Unſer iſt der Sieg!\\
  Dank ſey Gott! und Jeſus iſt Sieger!
\end{movement}

\begin{movement}{dortseh}
  \item[Tenore I]
  Dort ſeh ich aus den Toren Jeruſalems\\
  zween Schüler Jeſu gehn.\\
  In Zweifeln ganz, und ganz in Traurigkeit verloren,\\
  gehn ſie durch Wald und Feld,\\
  und klagen ihren Herrn.\\
  Der Herr geſellt ſich zu den Traurenden,\\
  umnebelt ihr Geſicht,\\
  hört ihre Zweifel an,\\
  giebt ihnen Unterricht:

  \item[Basso]
  Der Held aus Juda,\\
  dem die Völker dienen ſollen,\\
  muß erſt den Spott der Heyden,\\
  und ſeines Volks Verachtung leiden.\\
  Der mächtige Prophet von Worten und von Thaten\\
  muß durch den Freund, der mit ihm aß, verrathen,\\
  verworfen durch den andern Freund,\\
  verlaßen in der Noth von allen,\\
  den böſen Rotten in die Hände fallen.\\
  Es treten Frevler auf, und zeugen wider ihn:\\
  So ſpricht der Mund der Väter.\\
  Der König Iſraels verbirgt ſein Angeſicht\\
  vor Schmach und Speichel nicht.\\
  Er hält die Wangen ihren Streichen,\\
  den Rücken ihren Schlägen dar.\\
  Zur Schlachtbank hingeführt,\\
  thut er den Mund nicht auf.\\
  Gerechnet unter Miſſethäter,\\
  fleht er für ſie zu Gott hinauf.\\
  Durchgraben hat man ihn,\\
  an Hand und Fuß durchgraben.\\
  Mit Eßig tränkt man ihn in ſeinem großen Durſt,\\
  und miſchet Galle drein.\\
  Sie ſchütteln ihren Kopf um ihn.\\
  Er wird auf kurze Zeit von Gott verlaſſen ſeyn.\\
  Die Völker werden ſehn,\\
  wen ſie durchſtochen haben!\\
  Man theilet ſein Gewand,\\
  wirft um ſein Kleid das Loos.\\
  Er wird begraben, wie die Reichen.\\
  Und unverweſt am Fleiſch\\
  zieht Gott ihn aus dem Schooß der Erd hervor,\\
  und ſtellt ihn auf den Fels.\\
  Er gehet in ſeine Herrlichkeit\\
  zu ſeinem Vater ein.\\
  Sein Reich wird ewig ſein,\\
  ſein Name bleibt, ſo lange Mond und Sonne ſtehet.

  \item[Tenore I]
  Die Rede heilt der Freunde Schmerz.\\
  Mit Liebe wird ihr Herz zu dieſem Gaſt entzündet.\\
  Sie lagern ſich.\\
  Er bricht das Brodt, und ſaget Dank.\\
  Die Jünger kennen ſeinen Dank;\\
  der Nebel fällt, ſie ſehn ihn, er verſchwindet.
\end{movement}

\begin{movement}{willkommen}
  \item[Canto]
  Willkommen, Heyland! Freut euch, Väter!\\
  Die Hoffnung Zions iſt erfüllt!\\
  O dankt, ihr ungebohrnen Kinder!\\
  Gott nimmt für eine Welt voll Sünder\\
  ſein großes Opfer an.\\
  Der Heilige ſtirbt für Verräther:\\
  So wird des Richters Spruch erfüllt.\\
  Er tritt das Haupt der Hölle nieder,\\
  er bringet die Rebellen wieder:\\
  Der Himmel nimmt uns an.
\end{movement}

\begin{movement}{triumphb}\enlargethispage\baselineskip
  \item[Coro]
  Triumph! Triumph! der Fürſt des Lebens ſieget!\\
  Gefeßelt führt er Höll und Tod.\\
  Triumph! Triumph! die Siegesfahne flieget!\\
  Sein Kleid iſt noch vom Blute roth.
\end{movement}

\begin{movement}{eilf}
  \item[Tenore I]
  Eilf auserwählte Jünger,\\
  bey verſchloßnen Thüren,\\
  die Wut der Feinde ſcheuend,\\
  freuen ſich, daß Jeſus wieder lebt.

  \item[Tenore II]
  Ihr glaubt es, aber mich, erwidert Thomas,\\
  mich ſoll kein falſch Geſicht verführen.

  \item[Alto]
  Iſt er den Galiläerinnen nicht,\\
  auch dieſem Simon nicht erſchienen?\\
  Sahn ihn nicht Kleophas\\
  und ſein Gefährte dort bey Emmahus?

  \item[Canto]
  Ja, hier, mein Freund, an dieſem Ort\\
  ſahn wir ihn alle ſelbſt.

  \item[Tenore I]
  Es waren ſeine Mienen,\\
  die Worte waren ſeinen Worten gleich,\\
  er aß mit uns.

  \item[Tenore II]
  Betrogen hat man euch!\\
  Ihr ſelbſt, aus Sehnſucht,\\
  habt euch gern betrogen!\\
  Laßt mich ihn ſehn,\\
  mit allen Nägelmalen ſehn:\\
  Dann glaub auch ich,\\
  es ſey mein heißter Wunſch geſchehn.

  \item[Tenore I]
  Und nun zerfließt die Wolke,\\
  die den Herrn umzogen,\\
  der mitten unter ihnen ſteht,\\
  und ſpricht:

  \item[Basso]
  Der Friede Gottes ſey mit euch!\\
  Und du, Schwachgläubiger!\\
  Komm, ſiehe, zweifle nicht!

  \item[Tenore II]
  Mein Herr! mein Gott!\\
  ich ſeh, ich glaub, ich ſchweige.

  \item[Basso]
  So geh in alle Welt,\\
  und ſey mein Zeuge!
\end{movement}

\begin{movement}{meinherr}
  \item[Tenore II]
  Mein Herr! mein Gott! mein Herr! mein Gott!\\
  Dein iſt das Reich, die Macht iſt dein!\\
  Sowahr dein Fuß dieß Land betreten,\\
  wirſt du der Erde Schutzgott ſeyn.\\
  Jehovens Sohn wird ihn vertreten!\\
  Verſöhnte, kommt, ihn anzubeten!\\
  Erlöſte, ſagt ihm Dank!\\
  Zu dir ſteigt mein Geſang empor,\\
  aus jedem Thal, aus jedem Hain.\\
  Dir will ich auf dem Feld Altäre,\\
  und auf den Hügeln Tempel weyhn.\\
  Lallt meine Zunge nicht mehr Dank:\\
  ſo ſey der Ehrfurcht fromme Zähre\\
  mein letzter Lobgeſang.
\end{movement}

\begin{movement}{triumphc}
  \item[Coro]
  Triumph! Triumph! der Sohn des Höchſten ſieget!\\
  Er eilt zum Sühnaltar empor.\\
  Triumph! Triumph! Sein Vater iſt vergnüget.\\
  Er nimmt uns in der Engel Chor.
\end{movement}

\begin{movement}{aufeinem}
  \item[Tenore I]
  Auf einem Hügel, deſſen Rücken\\
  der Oelbaum und der Palmbaum ſchmücken,\\
  ſteht der Geſalbte Gottes.\\
  Um ihn ſtehn die ſeeligen Gefährten ſeiner Pilgrimſchaft.\\
  Sie ſehn erſtaunt von ſeinem Antlitz Strahlen gehn.\\
  Sie ſehn in einer lichten Wolke\\
  den Flammenwagen warten,\\
  der ihn führen ſoll.\\
  Sie beten an.\\
  Er hebt die Hände zum letzten Segen auf.

  \item[Basso]
  Seyd meines Geiſtes voll!\\
  Geht hin und lehrt,\\
  bis an der Erden Ende,\\
  was ihr von mir gehört:\\
  das ewige Gebot der Liebe!\\
  Gehet hin! thut meine Wunder!\\
  Gehet hin! Verkündigt allem Volke\\
  Verſöhnung, Friede, Seeligkeit.

  \item[Tenore II]
  Er ſagts, ſteigt auf,\\
  wird ſchnell empor getragen.\\
  Ein ſtrahlendes Gefolg\\
  umringet ſeinen Wagen.
\end{movement}

\clearpage
\begin{movement}{ihrthore}
  \item[Basso]
  Ihr Thore Gottes, öffnet euch!\\
  der König ziehet in ſein Reich.\\
  Macht Bahn, ihr Seraphinenchöre!\\
  Er ſteigt auf ſeines Vaters Thron.\\
  Triumph! werft eure Kronen nieder!\\
  Triumph! ſo ſchallt der weite Himmel wieder.\\
  Triumph! gebt unſrem Gott die Ehre!\\
  Heil unſrem Gott und ſeinem Sohn!
\end{movement}

\begin{movement}{gottfaehret}
  \item[Coro]
  Gott fähret auf mit Jauchzen!\\
  und der Herr mit heller Poſaune.\\
  Lobſinget, lobſinget Gott!\\
  Lobſinget, lobſinget unſerm Könige!
\end{movement}

\textlt{[\textit{Note on final music page:}]
Man hat nicht für nöthig befunden,
die übrigen aus den Pſalmen
beygefügten Verſe zu componiren,
weil das Stück doch lang genug iſt,
auch die letzte Arie ſchon
den Grad der Freude bezeichnet,
der der wahreſte Inhalt der
folgenden proſaiſchen Worte iſt.}
}

\eesScore

\end{document}
