\documentclass[parskip=full]{scrreprt}

\RedeclareSectionCommand[pagestyle=plain,afterskip=10pt plus 1pt]{chapter}
\setkomafont{chapter}{\mdseries\headingfont\fontsize{40}{40}\selectfont\color{black!80}}
\setkomafont{pageheadfoot}{\normalsize}

\def\pnumbox#1{#1\hspace*{8cm}}
\def\gobble#1{}
\DeclareTOCStyleEntry[
  indent=0pt,
  beforeskip=0pt,
  entryformat=\itshape,
  entrynumberformat=\textcolor{oldred},
  numwidth=2em,
  linefill=\hfill,
  pagenumberbox=\pnumbox,
  pagenumberformat=\itshape
]{tocline}{part}

\DeclareTOCStyleEntry[
  indent=0pt,
  beforeskip=0pt,
  entrynumberformat=\textcolor{oldred},
  numwidth=2em,
  linefill=\hfill,
  pagenumberbox=\pnumbox
]{tocline}{section}

\DeclareTOCStyleEntry[
  indent=0pt,
  beforeskip=-\parskip,
  entrynumberformat=\gobble,
  entryformat=\ltseries,
  numwidth=2em,
  linefill=\hfill,
  pagenumberbox=\pnumbox,
  pagenumberformat=\ltseries
]{tocline}{subsection}


\usepackage{polyglossia}
\setdefaultlanguage{english}


\usepackage{fontspec}
\setmainfont{Source Sans Pro}[
  ItalicFont = Source Sans Pro Italic,
  BoldFont = Source Sans Pro Bold,
  BoldItalicFont = Source Sans Pro Bold Italic,
  FontFace = {lt}{n}{Source Sans Pro Light},
  FontFace = {lt}{it}{Source Sans Pro Light Italic},
  FontFace = {sb}{n}{Source Sans Pro Semibold},
  FontFace = {sb}{it}{Source Sans Pro Semibold Italic},
  Numbers = Proportional,
  Ligatures = Common
]
\DeclareRobustCommand{\ltseries}{\fontseries{lt}\selectfont}
\DeclareRobustCommand{\sbseries}{\fontseries{sb}\selectfont}
\DeclareTextFontCommand{\textlt}{\ltseries}
\DeclareTextFontCommand{\textsb}{\sbseries}
\newfontfamily\headingfont{Fredericka the Great}


\usepackage[pass]{geometry}
\newgeometry{twoside,inner=20mm,outer=40mm,top=20mm,bottom=40mm}


\usepackage{url}
\urlstyle{same}


\usepackage{microtype}
\microtypesetup{verbose=silent}


\usepackage{booktabs,array,longtable}
\newcolumntype{L}[1]{%
  >{\raggedright\let\newline\\\arraybackslash\hspace{0pt}}p{#1}%
}


\usepackage{graphicx}


\usepackage{xcolor}
\definecolor{oldred}{rgb}{.8313,0,0}


\usepackage{pdfpages}


\usepackage{scrlayer-scrpage}
\pagestyle{scrheadings}
\clearscrheadfoot
\cfoot[\thepage]{\thepage}
\pagenumbering{roman}

\usepackage{enumitem}
\setlist[description]{%
 labelindent=2em,%
 labelwidth=6.5em,%
 leftmargin=8.5em,%
 labelsep=0pt,
 first=\ltseries,%
 font=\normalfont\itshape\ltseries%
}
\def\lyrefitem#1{\item[\lyref{#1}]}


\makeatletter

\def\@firstofthree#1#2#3{#1}
\def\@secondofthree#1#2#3{#2}
\def\@thirdofthree#1#2#3{#3}
\def\@firstoffour#1#2#3#4{#1}
\def\@secondoffour#1#2#3#4{#2}
\def\@thirdoffour#1#2#3#4{#3}
\def\@fourthoffour#1#2#3#4{#4}
\def\Dotfill{\leavevmode\cleaders\hb@xt@ .75em{\hss .\hss }\hfill \kern \z@}

\def\lyrefnumber#1{\expandafter\@setref\csname r@#1\endcsname\@firstofthree{#1}}
\def\lyreftitle#1{\expandafter\@setref\csname r@#1\endcsname\@secondofthree{#1}}
\def\lyrefpage#1{\expandafter\@setref\csname r@#1\endcsname\@thirdofthree{#1}}

\def\lyrefgenrenumber#1{\expandafter\@setref\csname r@#1\endcsname\@firstoffour{#1}}
\def\lyrefgenregenre#1{\expandafter\@setref\csname r@#1\endcsname\@secondoffour{#1}}
\def\lyrefgenretitle#1{\expandafter\@setref\csname r@#1\endcsname\@thirdoffour{#1}}
\def\lyrefgenrepage#1{\expandafter\@setref\csname r@#1\endcsname\@fourthoffour{#1}}

\def\lyref#1{%
  \begingroup%
  \makebox[0pt][l]{\color{oldred}\lyrefnumber{#1}}\hspace*{2em}%
  \lyreftitle{#1}\Dotfill\lyrefpage{#1}%
  \endgroup%
}
\def\lyrefpart#1{%
  \begingroup%
  \makebox[0pt][l]{\sbseries\color{oldred}\lyrefnumber{#1}}\hspace*{2em}%
  \makebox[0pt][l]{\sbseries\lyreftitle{#1}}\hspace*{6.5em}%
  \hfill\sbseries\lyrefpage{#1}%
  \endgroup%
}
\def\lyrefsection#1{%
  \begingroup%
  \makebox[0pt][l]{\color{oldred}\lyrefgenrenumber{#1}}\hspace*{2em}%
  \makebox[0pt][l]{\ltseries\lyrefgenregenre{#1}}\hspace*{6.5em}%
  \lyrefgenretitle{#1}\Dotfill\lyrefgenrepage{#1}%
  \endgroup%
}
\InputIfFileExists{../tmp/lilypond.ref}{}{\InputIfFileExists{../lilypond.ref}{}{}}


\newcommand\fancytitlehead{
  \headingfont%
  \fontsize{80}{80}\selectfont%
  \textcolor{black!80}{%
    \makebox[0pt][l]{\@ifundefined{@shortname}{\@lastname}{\@shortname}.}%
  }\\[15pt]%
  \fontsize{60}{60}\selectfont%
  \makebox[0pt][l]{\@ifundefined{@shorttitle}{\@title}{\@shorttitle}.}%
}


\def\firstname#1{\def\@firstname{#1}}
\def\lastname#1{\def\@lastname{#1}}
\def\shortname#1{\def\@shortname{#1}}
\def\shorttitle#1{\def\@shorttitle{#1}}
\def\namesuffix#1{\def\@namesuffix{#1}}
\def\scoring#1{\def\@scoring{#1}}
\def\parts#1{\def\@parts{#1}}

\firstname{\relax}
\lastname{\relax}
\namesuffix{\relax}
\scoring{\relax}
\parts{\relax}


\def\maketitle{%
\begin{titlepage}%
  \Large%
  {\@titlehead}%
  \vfill%
  {\strut\@firstname}\\%
  {\sbseries\color{oldred}\strut\@lastname}\\%
  {\strut\@namesuffix}%
  \vfill%
  {\sbseries\@title}\\%
  {\@subtitle}\\[\baselineskip]%
  {\itshape\@scoring}%
  \vfill%
  {\itshape\@parts}\hspace*{\fill}\raisebox{0pt}[0pt][0pt]{\includegraphics{ees_logo}}%
\end{titlepage}%
}


\newif\iftemplate\templatetrue
\newif\ifprintreport\printreportfalse
\directlua{
scores = {
  fl1 = "Flauto I",
  fl2 = "Flauto II",
  ob1 = "Oboe I",
  ob2 = "Oboe II",
  cor = "Corno I, II in B",
  ottoni = "Clarino I, II in C\string\\newline Timpani in C–G",
  vl1 = "Violino I",
  vl2 = "Violino II",
  vla = "Viola",
  coro = "Coro",
  org = "Organo",
  b = "Bassi",
  full_score = "Full Score"
}

last_arg = arg[\string#arg]
texio.write("Last argument: " .. last_arg)
if not (scores[last_arg] == nil) then
  tex.print("\string\\def\string\\lypdfname{" .. last_arg .. "}")
  tex.print("\string\\parts{" .. scores[last_arg] .. "}")
  if (last_arg == "full_score") then
    tex.print("\string\\printreporttrue")
  end
end
}

\@ifundefined{lypdfname}{%
  \templatefalse
  \printreporttrue
  \parts{Draft}
}{\templatetrue}

\makeatother



\begin{document}
\frenchspacing

\titlehead{\fancytitlehead}
\firstname{Johann Adolph}
\lastname{Scheibe}
\title{Die Auferstehung und Himmelfarth Jesu}
\shorttitle{Die Auferstehung}
\subtitle{Auferstehungs- und Himmelfarths=Cantate\\SchW online B2:301, 302}
\scoring{S, A, T, B (solo), S, A, T, B (coro), 2 fl, 2 ob, 2 cor, 2 clno, timp, 2 vl, vla, b, org}
\maketitle


\thispagestyle{empty}

\vspace*{\fill}

\raisebox{-4mm}{\includegraphics{byncsaeu}}\hspace*{1em}Wolfgang Esser-Skala, 2021

© 2021 by Wolfgang Esser-Skala. This edition is licensed under the Creative Commons Attribution-NonCommercial-ShareAlike 4.0 International License. To view a copy of this license, visit \url{http://creativecommons.org/licenses/by-nc-sa/4.0/}.

Music engraving by LilyPond 2.22.0 (\url{http://www.lilypond.org}).\\
Front matter typeset with Source Sans Pro and Fredericka the Great.

\textit{First version, June 2021}

\vspace*{2cm}

\ifprintreport
\chapter*{Critical Report}

This edition bases upon two autograph manuscripts in the Staatsbibliothek Berlin (D-B). The digital versions of the manuscripts are available at
\begin{enumerate}
  \item \url{http://resolver.staatsbibliothek-berlin.de/SBB0001D7ED00000000}\\
  (siglum Mus.ms.autogr. Scheibe, J. A. 1a; see also RISM ID 464131750), and
  \item \url{http://resolver.staatsbibliothek-berlin.de/SBB0001D7F000000000}\\
  (siglum Mus.ms.autogr. Scheibe, J. A. 1b; see also RISM ID 464131751)
\end{enumerate}

In general, this edition closely follows the manuscript. Any changes that were introduced by the editor are indicated by italic type (lyrics, dynamics and directions), parentheses (expressive marks and bass figures) or dashes (slurs and ties). Accidentals are used according to modern conventions. Asterisks denote changes that are clarified in the detailed remarks below.\footnote{Abbreviations: A, alto; B, bass; b, basses; fl, flute; clno, clarion; cor, horn; Ms, manuscript; ob, oboe; org, organ; r,~rest; S, soprano; T, tenor; timp, timpani; vl, violin; vla, viola.}

\bigskip

\begin{longtable}{lll L{10cm}}
  \toprule
  \itshape Mov. & \itshape Bar & \itshape Staff & \itshape Note \\
  \midrule \endhead
  – & –  & A       & S 2 may be replaced by A throughout the piece (as indicated by \textit{Canto 2 overo Alto} in all respective movements). \\
  1 & 73 & ob 1, 2 & grace note missing in Ms \\
  \bottomrule
\end{longtable}


This edition has been compiled and checked with utmost diligence. Nevertheless, errors and mistakes cannot be totally excluded. Please report any errors and mistakes to \url{wolfgang@esser-skala.at} or create an issue or pull request on the edition’s GitHub page \url{https://github.com/skafdasschaf/scheibe-die-auferstehung}. Your help will be greatly appreciated.

\bigskip
\textit{Koppl, June 2021\\
Wolfgang Esser-Skala}

\cleardoublepage
\chapter*{Contents.}


\lyrefsection{gottduwirst}

\begin{description}
  \item[Coro]
  Gott! du wirſt ſeine Seele nicht in der Hölle laſſen!\\
  Und nicht zugeben, daß dein Heiliger die Verweſung ſehe!
\end{description}

\lyrefsection{judaeazittert}

\begin{description}
  \item[Tenore I]
  Judäa zittert! ſeine Berge beben!\\
  der Jordan flieht den Strand!\\
  Was zitterſt du, Judäens Land?\\
  Ihr Berge, warum bebt ihr ſo?\\
  Was war dir, Jordan, daß dein Strom zurücke floß?\\
  Der Herr der Erde steigt empor aus ihrem Schooß,\\
  tritt auf den Fels, und zeigt der ſtaunenden Natur ſein Leben.\\
  Des Himmels Myriaden liegen auf der Luft rings um ihn her;\\
  und Cherub Michael fährt nieder,\\
  und rollt des vorgeworfnen Steines Laſt\\
  hinweg von ſeines Königs Gruft.\\
  Sein Antlitz flammt, ſein Auge glühet.\\
  Die Schaar der Römer ſtürzt erblaßt auf ihre Schilde:\\
  Flieht, ihr Brüder! der Götter Rache trifft uns! fliehet!
\end{description}

\lyrefsection{meingeist}

\begin{description}
  \item[Tenore I]
  Mein Geiſt voll Furcht und Freude bebet:\\
  der Fels zerſpringt! die Nacht wird lichte!\\
  Seht, wie er auf den Lüften ſchwebet!\\
  Seht, wie von ſeinem Angeſichte\\
  die Glorie der Gottheit ſtrahlt.\\
  Rang Jeſus nicht mit tauſend Schmerzen?\\
  Empfing ſein Gott nicht ſeine Seele?\\
  Floß nicht ſein Blut aus ſeinem Herzen?\\
  Hat nicht der Held in dieſer Hölle\\
  der Erde ſeine Schuld bezahlt?
\end{description}

\lyrefsection{triumph}

\begin{description}
  \item[Coro]
  Triumph! Triumph! des Herrn Geſalbter ſieget!\\
  Er ſteigt aus ſeiner Felſengruft.\\
  Triumph! Triumph! ein Chor von Engeln flieget\\
  mit lautem Jubel durch die Luft.
\end{description}

\lyrefsection{diefrommen}

\begin{description}
  \item[Tenore I]
  Die frommen Töchter Zions gehn verwundernd\\
  durch des offnen Grabes Thür;\\
  und ſchaudernd fahren ſie zurück.\\
  Sie ſehn, in Glanz gehüllt, den Boten des Ewigen,\\
  der freundlich ſpricht:

  \item[Canto]
  Entſetzt euch nicht!\\
  Ich weiß, ihr suchet euren Todten,\\
  den Nazaräer Jeſus hier,\\
  daß ihr ihn ſalbt, daß ihr ihn klagt.\\
  Hier iſt er nicht vorhanden.\\
  Er hat es euch zuvor geſagt:\\
  Er lebt! Er lebt! Er iſt erſtanden!
\end{description}

\lyrefsection{seygegruesset}

\begin{description}
  \item[Alto]
  Sey gegrüßet, Fürſt des Lebens!\\
  Jauchzet, die ſein Tod betrübte!\\
  Er, den dieſer Hügel deckte,\\
  er, Jeſus, lebt!\\
  Ihr klagt vergebens!\\
  Sehet da, ſein leeres Grab!\\
  Der die Todten auferweckte,\\
  ſollte der im Grabe bleiben?\\
  Himmel! Soll der Gottgeliebte,\\
  ſoll der Gottheit Sohn zerſtäuben?\\
  Todesengel laſſet ab!
\end{description}

\cleardoublepage
\fi

\iftemplate
\includepdf[pages=-]{../tmp/\lypdfname.pdf}
\fi

\end{document}
